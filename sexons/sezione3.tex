% --- Contenuto LaTeX autogenerato da sezioneIII.md (sezione 4) ---

\section{Le Patologie dei Nastri Magnetici}
\subsection{Il degrado naturale e le correnti magnetiche}
Il degrado dei nastri magnetici è un processo inevitabile causato da molteplici fattori. La nostra realtà è attraversata da correnti magnetiche di ogni tipo che, nel tempo, causano una progressiva demagnetizzazione del segnale registrato. Questo processo trasforma gradualmente l'informazione registrata in rumore di fondo.

L'esperienza diretta con i nastri di Scelsi ha permesso di osservare come questo degrado ''naturale'' sia spesso il minore dei problemi. La qualità iniziale delle registrazioni domestiche, con il loro alto livello di rumore ambientale e distorsione, rende il deterioramento magnetico relativamente poco significativo rispetto ad altre forme di degrado fisico-chimico\cite[p. 170]{Bernardini2007recoveringgia}.
\subsection{La sindrome degli Ampex 456}
Il caso più drammatico di patologia specifica riguarda i nastri Ampex 456, considerati all'epoca della loro introduzione negli anni Sessanta come prodotti di altissima qualità per uso professionale. Durante il laboratorio ho potuto osservare direttamente le conseguenze della ''sticky shed syndrome'' che affligge questi nastri: il legante poliuretanico utilizzato nella formulazione subisce un processo di idrolisi che causa il distacco progressivo dello strato magnetico dal supporto.

Il trattamento di questi nastri richiede una procedura specializzata che ho potuto seguire in collaborazione con i tecnici della Discoteca di Stato:

\begin{enumerate}
    \item \textbf{Diagnosi preliminare}: identificazione visiva e tattile della patologia attraverso l'osservazione di residui sulla flangia e test di scorrimento manuale.
\end{enumerate}

\begin{enumerate}
    \item \textbf{Trattamento termico}: i nastri vengono collocati in forni speciali che mantengono una temperatura costante tra 49 e 53 gradi Celsius con precisione di ±0,5°C per un periodo di 4-4,5 ore.
\end{enumerate}

\begin{enumerate}
    \item \textbf{Trasferimento immediato}: il nastro deve essere digitalizzato mentre è ancora caldo, quando il legante temporaneamente consolidato dal trattamento termico mantiene la sua coesione.
\end{enumerate}

\begin{enumerate}
    \item \textbf{Singola opportunità}: come osservato durante le sessioni, questi nastri possono generalmente essere letti una sola volta dopo il trattamento. Un secondo passaggio risulta impossibile poiché il nastro lascia sul percorso della testina uno strato di ossido magnetico che appare come polvere nera e perde quasi interamente tutte le informazioni contenute in esso.
\end{enumerate}
\subsection{Muffe e contaminazioni biologiche}
Circa il 2-3% dei nastri dell'archivio Scelsi presenta contaminazioni da muffe, corrispondente a circa 10-15 nastri sul totale della collezione. La presenza di muffe è stata favorita dalle condizioni di conservazione non ottimali, particolarmente nei casi in cui i nastri erano stati conservati nelle buste di plastica che trattenevano l'umidità.

Il trattamento delle muffe, osservato durante una sessione specializzata, segue un protocollo preciso:

\begin{enumerate}
    \item \textbf{Pre-trattamento termico}: i nastri ammuffiti vengono sottoposti a un ciclo di ''cottura'' breve (circa un'ora) a temperatura moderata per essiccare la muffa e trasformarla in polvere.
\end{enumerate}

\begin{enumerate}
    \item \textbf{Aspirazione controllata}: utilizzando un aspiratore speciale con filtrazione HEPA, la muffa essiccata viene rimossa delicatamente dalle spire del nastro.
\end{enumerate}

\begin{enumerate}
    \item \textbf{Pulizia del percorso}: durante la digitalizzazione, le testine devono essere pulite frequentemente per rimuovere residui di muffa che potrebbero compromettere la lettura.
\end{enumerate}
\subsection{Adesione delle spire e problematiche meccaniche}
Un problema frequente riscontrato durante il laboratorio riguarda l'adesione delle spire, fenomeno per cui gli strati adiacenti del nastro si incollano tra loro. Questa patologia può avere diverse cause: degradazione del legante, pressione eccessiva durante lo stoccaggio, variazioni di temperatura e umidità.

Il trattamento più singolare osservato durante le sessioni riguarda l'uso del borotalco per facilitare lo scorrimento delle spire adesive. La procedura, apparentemente rudimentale ma efficace, prevede l'applicazione manuale del borotalco sulla costa del nastro mentre questo scorre lentamente. Il talco agisce come lubrificante secco, permettendo la separazione delle spire senza danneggiare lo strato magnetico. Particolarmente interessante è l'osservazione che il cigolio prodotto dalle spire adesive non è solo un sintomo del problema, ma entra effettivamente nella registrazione audio durante il trasferimento, richiedendo quindi un intervento preventivo.
\subsection{L'effetto pre-eco}
Durante l'analisi dei nastri è emerso un fenomeno particolare noto come effetto pre-eco, causato dal trasferimento magnetico tra spire adiacenti. Nei nastri conservati ''in testa'' (con l'inizio del nastro all'esterno della bobina), il segnale delle spire interne può imprimere una debole copia di sé sulle spire esterne, creando un'eco che precede il segnale principale.

Questo fenomeno, udibile in alcune registrazioni commerciali dell'epoca come il secondo album dei Led Zeppelin citato durante le sessioni, ha importanti implicazioni per la conservazione. La pratica standard prevede infatti la conservazione dei nastri ''in coda'' per minimizzare questo effetto. Tuttavia, la maggior parte dei nastri di Scelsi è stata trovata conservata ''in testa'', richiedendo una procedura di doppio riavvolgimento prima della digitalizzazione:

\begin{enumerate}
    \item Avvolgimento rapido su flangia temporanea
    \item Secondo avvolgimento su altra flangia temporanea  
    \item Trasferimento digitale con riavvolgimento finale sulla flangia originale
\end{enumerate}

Questa procedura, pur generando stress meccanico sul nastro, è necessaria per garantire la corretta conservazione post-digitalizzazione.
\subsection{Il deterioramento delle giunture}
Le giunture originali presenti sui nastri rappresentano punti di particolare fragilità. Durante il laboratorio ho potuto osservare come lo scotch utilizzato per le giunture negli anni Cinquanta-Settanta sia soggetto a un processo di degrado che lo rende fragile e privo di adesività. In diversi casi, le giunture si sono separate durante la manipolazione del nastro, richiedendo un intervento di riparazione immediato.

La procedura di rifacimento delle giunture richiede precisione e materiali appropriati:
- Rimozione completa della giuntura originale degradata
- Allineamento preciso delle estremità del nastro
- Applicazione di nuovo nastro adesivo specifico per giunture audio
- Verifica della tenuta meccanica prima del passaggio sulle testine

Fortunatamente, come notato durante le sessioni, i nastri di Scelsi presentano relativamente poche giunture rispetto ai nastri provenienti da studi di musica elettronica, dove la pratica del montaggio creava centinaia di giunture per nastro.

