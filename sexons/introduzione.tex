% --- Contenuto LaTeX autogenerato da introduzione.md (sezione 1) ---

\section{INTRODUZIONE}
Durante le cinquanta ore di laboratorio svolte presso la Fondazione Isabella Scelsi sotto la supervisione del \personaViva{M° Nicola Bernardini}{1956}, ho avuto l'opportunità di partecipare attivamente a uno dei progetti di recupero e digitalizzazione più complessi e affascinanti nel campo del restauro audio contemporaneo: il salvataggio dell'intero corpus di nastri magnetici del compositore \persona{Giacinto Scelsi}{1905}{1988}.

Il progetto, iniziato nel 2006 e tuttora in corso, rappresenta un caso paradigmatico delle sfide che il restauro audio deve affrontare quando i documenti sonori non sono semplicemente registrazioni da preservare, ma costituiscono parte integrante del processo creativo di un compositore. Come documentato da Bernardini, \citazionebreve{Right from the start it was clear that this would not have been the classical audio recover–and–restore job}{\cite[p. 2]{Bernardini2007recoveringgia}}, poiché Scelsi utilizzava i nastri come veri e propri ''sketchpad'' compositivi, registrando improvvisazioni che sarebbero poi state trascritte in partitura da suoi collaboratori.

L'archivio conta in tutale 749 nastri magnetici, di cui 278 classificati come ''di primaria importanza'' perché contenenti le improvvisazioni originali del compositore\cite{Bernardini2012themul}. A questi si aggiungono 76 dischi in alluminio laccato recentemente scoperti, che testimoniano come Scelsi avesse iniziato a utilizzare la registrazione come strumento compositivo ancor prima dell'avvento del nastro magnetico\cite[p. 185]{Bernardini2012themul}.

La peculiarità di questo progetto risiede nella necessità di preservare non solo il contenuto musicale, ma anche quello che normalmente verrebbe considerato ''rumore'': i suoni ambientali, i click dei registratori, persino il riverbero delle stanze si sono rivelati informazioni preziose per la datazione e contestualizzazione dei materiali. Durante il laboratorio ho potuto constatare come, paradossalmente, \citazionebreve{the electrical noise added by time degradation the tapes is insignificant compared to the large amount of other noise}{\cite[p. 170]{Bernardini2012themul}} presente nelle registrazioni originali.

Il presente lavoro documenta le tecniche e metodologie applicate in questo progetto, organizzandosi in quattro parti principali: la prima dedicata alla descrizione del patrimonio sonoro e del metodo compositivo di Scelsi; la seconda alle problematiche conservative specifiche dei nastri magnetici e alle tecniche di restauro adottate; la terza al processo di digitalizzazione e post-produzione; la quarta alle prospettive di ricerca musicologica che questo lavoro di recupero ha aperto.

L'esperienza diretta con i nastri mi ha permesso di comprendere come il restauro audio, quando applicato a documenti di processo creativo, richieda un approccio multidisciplinare che integri competenze tecniche, sensibilità musicologica e comprensione del contesto storico-culturale. In questo senso, il progetto Scelsi rappresenta un laboratorio ideale per esplorare le frontiere contemporanee del restauro audio digitale.

