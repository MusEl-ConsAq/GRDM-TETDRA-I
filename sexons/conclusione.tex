Il progetto di recupero e digitalizzazione dell'archivio sonoro di Giacinto Scelsi, condotto presso la Fondazione Isabella Scelsi sotto la supervisione del M° Nicola Bernardini, rappresenta un'impresa di notevole complessità e importanza nel campo del restauro audio e della musicologia contemporanea. L'esperienza diretta con i 749 nastri magnetici e i 76 dischi in alluminio laccato ha permesso di comprendere a fondo non solo le sfide tecniche legate alla conservazione di supporti fragili e spesso degradati, ma anche le peculiarità di un processo creativo unico, in cui la registrazione era parte integrante della composizione stessa.

L'analisi dell'archivio ha rivelato la necessità di un approccio multidisciplinare che superasse la semplice distinzione tra nastri di "primaria" e "secondaria" importanza, data la frequente commistione di materiali. La comprensione del metodo compositivo di Scelsi, basato sull'improvvisazione registrata e sulla successiva mediazione dei trascrittori (Tosatti, Cafaro, Filippini), è stata arricchita dallo studio delle condizioni di registrazione domestica, dall'analisi degli strumenti (incluse le ondioline) e delle velocità di registrazione variabili, che testimoniano un uso sperimentale e non convenzionale del mezzo.

Le patologie riscontrate sui nastri – dalla "sticky shed syndrome" alle muffe, dall'adesione delle spire all'effetto pre-eco – hanno richiesto l'applicazione di protocolli di restauro specifici, spesso condotti in collaborazione con istituzioni specializzate come la Discoteca di Stato. Questi interventi, sebbene mirati alla stabilizzazione fisica dei supporti, sono stati guidati dalla filosofia di preservare l'integrità documentale del materiale, inclusi i "rumori" contestuali che si sono rivelati preziosi per la datazione e la comprensione.

La catena di trasferimento digitale, implementata con standard professionali (Studer A800, Pro Tools a 96kHz/24bit), è stata adattata per gestire le configurazioni non standard dei nastri di Scelsi, come le velocità multiple e le registrazioni bidirezionali. La scelta di non intervenire con restauri invasivi sul segnale audio, ma di concentrarsi sulla catalogazione meticolosa dei contenuti e sulla documentazione fotografica, riflette l'obiettivo primario del progetto: rendere accessibile ai ricercatori un corpus sonoro il più fedele possibile all'originale, completo di tutte le sue "imperfezioni" significanti.

Questo lavoro di recupero non è solo un atto di conservazione, ma apre nuove prospettive per la ricerca musicologica. La disponibilità dell'archivio digitalizzato permetterà studi approfonditi sul processo creativo di Scelsi, sulla relazione tra improvvisazione e partitura, sul ruolo dei trascrittori e sull'evoluzione del suo linguaggio. L'esperienza condotta dimostra come il restauro audio, applicato a documenti di processo creativo, diventi esso stesso uno strumento di indagine musicologica, capace di illuminare aspetti inediti dell'opera di uno dei compositori più enigmatici del Novecento e di esplorare le frontiere del restauro audio digitale.